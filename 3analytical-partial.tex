\chapter{解析的に偏微分方程式を解く}
\label{analytical-partial}
さて、常微分方程式の次は偏微分方程式です。偏微分方程式はその名の通り式に偏微分した関数が出てくる形の微分方程式です。これはあまりにも複雑で、人間が解けることはほとんどありません。しかし、一部の場合において解析的に解くことが可能です。本章では解析的に解ける偏微分方程式を解説していきます。







\section{変数分離形}
\label{partial-separate}
偏微分方程式でも変数分離が使える場合があります。関数$u$を求める微分方程式

\begin{eqnarray}
    f(x)\frac{\partial u}{\partial x}=g(y)\frac{\partial u}{\partial y}
    \label{eq:partial-separate}
\end{eqnarray}

\noindent
を考えてみましょう。この微分方程式について、もし

\begin{eqnarray}
    \frac{\partial u}{\partial x}=\mbox{($y$を含まない関数)} \\
    \frac{\partial u}{\partial y}=\mbox{($x$を含まない関数)}
\end{eqnarray}

\noindent
であるとすると、式\ref{eq:partial-separate}は、

\begin{itemize}
    \item 左辺は$x$のみの関数
    \item 右辺は$y$のみの関数
\end{itemize}

\noindent
となっています。「変数が分離」していますね。左辺は$x$、右辺は$y$の関数で、この両辺が恒等的に等しいのです。つまり、この両辺はそれぞれ定数となります。その定数を$C$とすれば、

\begin{eqnarray}
    f(x)\frac{\partial u}{\partial x}&=&C \\
    \frac{\partial u}{\partial x}&=&\frac{C}{f(x)} \nonumber \\
    u(x,y)&=&\int\frac{C}{f(x)}\dd x+h(y) \setcounter{equation}{5}
    \label{eq:pqrtial-separate-1}
\end{eqnarray}

\noindent
と、$u$を$f(x)$で表すことができました。ここで、積分定数と同じ役割の$h(y)$は$y$の関数であることに注意してください。ここから$h(y)$を求める作業に入ります。

求めた$u$を偏微分して$\partial u/\partial y$を求めてみましょう。式\ref{eq:pqrtial-separate-1}の右辺第一項は$x$のみの関数なので、

\begin{eqnarray}
    \frac{\partial u}{\partial y}=\frac{\dd}{\dd y}h(y)
\end{eqnarray}

\noindent
です。元の微分方程式\ref{eq:partial-separate}より、

\begin{eqnarray}
    g(y)\frac{\partial u}{\partial y}=C \\
    \frac{\partial u}{\partial y}=\frac{C}{g(y)}
\end{eqnarray}

\noindent
より、$h(y)$をこの右辺の積分で得られ、結局、

\begin{eqnarray}
    u(x,y)=\int\frac{C}{f(x)}\dd x+\int\frac{C}{g(y)}\dd y
\end{eqnarray}

\noindent
として解$u(x,y)$が求まります。








\section{特性曲線}
\label{char-curve}
次に、このような形の偏微分方程式を考えてみましょう。

\begin{eqnarray}
    P(x,y)\frac{\partial u}{\partial x}+Q(x,y)\frac{\partial u}{\partial y}=0
\end{eqnarray}

これを満たす関数$u_0(x,y)$を一つ見つけられたなら、任意の関数$F(t)$について$t=u_0$とした関数$F(u_0)$もこの微分方程式を満たします。このとき、どの関数$F(u_0)$ともなっていない$u_0$を特性曲線と言います。









\section{演算子の因数分解}
\label{pqrtial-factorization}
常微分方程式と同様に演算子を因数分解できる場合があります。

\begin{eqnarray}
    \pqty{\frac{\partial^2}{\partial x^2}+(a+b)\frac{\partial^2}{\partial x\partial y}+ab\frac{\partial^2}{\partial y^2}}u(x,y)=0
\end{eqnarray}

\noindent
という微分方程式を

\begin{eqnarray}
    \pqty{\frac{\partial}{\partial x}+a\frac{\partial}{\partial y}}\pqty{\frac{\partial}{\partial y}+b\frac{\partial}{\partial x}}u(x,y)=0
\end{eqnarray}

\noindent
と因数分解してみましょう。常微分方程式と同様、

\begin{eqnarray}
    \pqty{\frac{\partial}{\partial x}+a\frac{\partial}{\partial y}}u=0\ \mbox{または}\ \pqty{\frac{\partial}{\partial y}+b\frac{\partial}{\partial x}}u=0
\end{eqnarray}

\noindent
として一階の偏微分方程式に分解できます。




\clearpage